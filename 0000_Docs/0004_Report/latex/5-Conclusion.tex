\section{Bilan}
Le cahier des charges a été respecté dans son intégralité. Le mode veille
fonctionne consomme $9[\mu A]$.
Le gameplay est proche du vrai Pong, l'utilisation du stylet rend le gameplay
plus agréable qu'au doigt.
Le réglage du rétroéclairage fonctionne parfaitement, et permet aussi meilleure
autonomie.

\section{Points à corriger}
\begin{itemize}
  \item Sur le PCB v1.0, les signaux PGC et PGD ont été inversés sur le port
  de programmation.
  \item Sur le PCB v1.0 Les connecteurs de piles sont trop proches l'un de l'autre, du scotch.
  isolant a dû être placé pour éviter des courts-circuits.
  \item Sur le boîtier v2.1, les fixations de la plaque supérieur sont trop
  fragiles.
  \item Sur le boîtier v2.1, l'accès aux piles est trop étroit et rend
  le changement de pile avec les doigts très désagréable.
  \item Sur le boîtier v2.1, le stylet ne peut pas rentrer correctement dans
  le boîtier à cause de l'impression 3D imprécise.
  \item Les pull-downs présentent sur les Gate des MOSFET ne sont pas
  nécessaires. Le PIC18LF25K22, lorsqu'il part en veille, concerve
  l'état de ses entrées/sorties d'avant veille.
\end{itemize}

\section{Voies d'amélioration}
\begin{itemize}
  \item Ajout d'un joystick ou slider pour le contrôle du Paddle.
  \item Amélioration du menu du PongBoy, ajout d'informations visuelles pour
  l'utilisateur (par ex: le slider de luminosité n'a pas de légende).
  \item Ajout de plusieurs réglages dans le menu paramètres (par ex: choisir
  de jouer sur fond blanc plutôt que fond noir).
  \item Amélioration du gameplay (balle qui s'accélère durant la manche).
  \item Ajout d'un choix de difficulté de l'adversaire dans Pong.
  \item Implémentation du haut-parleur, ajout de feedback sonore dans le menu
  et dans le Pong.
  \item Implémentation du capteur de luminosité ambiante, et réglage automatique
  du rétroéclairage.
  \item Implémentation du mode multi-joueur.
\end{itemize}

\section{Signatures}
Samy Francelet
