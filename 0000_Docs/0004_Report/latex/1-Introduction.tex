\begin{summary}
\section{Contexte}
Ce projet a été réalisé dans le cadre du cours
\emph{Summer School 2 - Infotronic} de la Haute école d'ingénieur du Valais.
Ce cours de 3 semaines se passe à la fin de l'été, avant la reprise de
septembre. Il a pour but de mettre en pratique les compétences acquises
durant la 2e année du bachelor.\\
L'objectif du projet est de développer une console de jeu portable,
avec écran tactile, permettant de jouer au jeu \emph{Pong}.\\
Normalement, des équipes de 2 personnes sont formées, mais
dû au fait que nous sommes un nombre impaire, je me suis retrouvé à développer
le PongBoy seul.
\end{summary}

\section{Spécifications}
\begin{itemize}
  \item La console doit être alimentée par 2 piles AAA en série, et ainsi
  fonctionner en $3[V]$.
  \item Une protection contre l'inversion de polarité des piles doit être
  présente.
  \item La console doit pouvoir être mise en veille ultra-basse-consommation,
  pour éviter de changer les piles AAA trop régulièrement.
  \item L'écran tactile résistif NHD-2.4-240320SF-CTXL-FTN1 doit être utilisé,
  et une interface tactile doit être implémentée.
  \item L'utilisation d'une puce driver pour l'écran tactile est prohibée,
  une librairie driver est fournie.
  \item Le micro-controlleur PIC18LF25K22 doit être utilisé, au format SOIC,
  pour ses capacités d'ultra-basse-consommation.
  \item L'alimentation de l'écran LCD ainsi que de son rétro-éclairage doit
  pouvoir être coupée pour respecter la contrainte d'ultra-basse-consommation.
\end{itemize}
